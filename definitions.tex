\begin{frame}

\frametitle{Основные определения}

\begin{itemize}
\item \emph{Домен} --- множество допустимых значений некоторого типа данных.

Например, домен ``Номер зачетной книжки'' может быть подмножеством
множества целых чисел.

%% Данные считаются сравнимыми только в том случае, когда они относятся к 
%% одному домену.

\item \emph{Отношение} --- подмножество декартового произведения некоторого
набор доменов. Пусть $D_1, D_2, \dots D_n$ --- исходный набор доменов.
Тогда, $R \subseteq D_1 \times D_2 \times\dots\times D_n$ является
отношением.

\item \emph{Атрибут} --- название (имя) домена, входящего в отношение.

\item \emph{Степень отношения} (или \emph{арность}) --- количество атрибутов,
входящих в отношение.

\item \emph{Кортеж} --- элемент отношения (упорядоченный набор значений доменов).

\item \emph{Кардинальность} (или \emph{кардинальное число отношения}) --- количество
кортежей отношения.
\end{itemize}

\end{frame}

