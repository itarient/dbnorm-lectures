\begin{frame}

\frametitle{Первая нормальная форма}

\emph{Отношение находится в первой нормальной форме (1НФ)} тогда и только тогда,
когда каждый кортеж отношения содержит только одно значение для
каждого атрибута.

\medskip

Примечания:
\begin{enumerate}
\item \underline{В теории}: отношение всегда находится в 1НФ по определению самого
понятие \emph{отношение}.
\item \underline{На практике}: Таблица не всегда может быть правильным представлением
отношения, а значит может не находиться в 1НФ.
\end{enumerate}

\medskip

Пример:

\begin{tabbing}
Ненормализованная таблица\hspace{1cm} \= Нормализованная таблица \\

\begin{tabular}{|l|l|}\hline
	ФИО & Номер телефона \\	\hline
	\multirow{2}{*}{Иванов И.И.} & 111-11-11 \\
	                             & 222-22-22 \\	\hline
	Петров П.П.                  & 333-33-33 \\	\hline	
\end{tabular}
\>
\begin{tabular}{|l|l|}\hline
	ФИО & Номер телефона \\ \hline
	Иванов И.И. & 111-11-11 \\ \hline
	Иванов И.И. & 222-22-22 \\ \hline
	Петров П.П. & 333-33-33 \\\hline	
\end{tabular}
\end{tabbing}

\end{frame}

